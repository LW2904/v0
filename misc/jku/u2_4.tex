\documentclass{article}
\usepackage[utf8]{inputenc}
\usepackage{enumerate}

\title{Übung 2}
\author{Laurenz Weixlbaumer, 11804751}
\date{Oktober 2018}

\renewcommand{\arraystretch}{1.3}
\renewcommand\thesubsection{(\alph{subsection})}

\usepackage{mathtools}

\begin{document}

\maketitle

\stepcounter{section}\stepcounter{section}\stepcounter{section} % Start at 4
\section{Grundrechnungsarten}

\begin{enumerate}[{(\alph{enumi})}]
\item Addiere folgende Binärzahlen: $11110_2$ und $100111_2$.

\begin{center}
\begin{tabular}{l | c | c | c | c | c | c | c }
     Summand 1 & 0 & 0 & 1 & 1 & 1 & 1 & 0 \\
     \hline
     Summand 2 & 0 & 1 & 0 & 0 & 1 & 1 & 1 \\
     \hline
     Übertrag  & 1 & 1 & 1 & 1 & 1 & 0 & 0 \\
     \hline
     \hline
     Summe     & 1 & 0 & 0 & 0 & 1 & 0 & 1
\end{tabular}
\end{center}

\item Addiere folgende Oktalzahlen: $23_8$ und $17_8$.

\begin{center}
\begin{tabular}{l | c | c }
     Summand 1 & 2 & 3 \\
     \hline
     Summand 2 & 1 & 7 \\
     \hline
     Übertrag  & 1 & 0 \\
     \hline\hline
     Summe     & 4 & 2
\end{tabular}
\end{center}

\item Multipliziere folgende Binärzahlen: $110011_2$ und $1110_2$.

\begin{center}
\begin{tabular}{ c c c c c c c c c c c c }
      & 1 & 1 & 0 & 1 & 1 & * & 1 & 1 & 1 & 0 \\
    \hline
      & 1 & 1 & 0 & 1 & 1 &   &   &   &   & + \\
      &   & 1 & 1 & 0 & 1 & 1 &   &   &   & + \\
      &   &   & 1 & 1 & 0 & 1 & 1 &   &   & + \\
      &   &   &   & 0 & 0 & 0 & 0 & 0 &   & + \\
    \hline\hline
    1 & 0 & 1 & 1 & 1 & 1 & 0 & 1 & 0 &   & = \\
\end{tabular}
\end{center}

\clearpage

\item Multipliziere folgende Hexadezimalzahlen: $AB_{16}$ und $CD_{16}$.

\begin{center}
\begin{tabular}{ c c c c c }
    A & B & * & C & D \\
    \hline
    8 & 0 & 4 &   & +\\
      & 8 & A & F & + \\
    \hline\hline
    8 & 8 & E & F & =
\end{tabular}
\end{center}

\end{enumerate}

\end{document}

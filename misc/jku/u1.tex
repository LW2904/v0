\documentclass{article}
\usepackage[utf8]{inputenc}

\title{\vspace{-2ex}Übung 1}
\author{Laurenz Weixlbaumer, 11804751}
\date{October 2018}

\renewcommand\thesubsection{(\alph{subsection})}

\usepackage{mathtools}

\begin{document}

\maketitle

\section{Wahrheitstabellen}
Überprüfe mittels einer vollständigen Wahrheitstabelle die (Un)-Gleichheit folgender Aussagen.

\subsection{$x \lor y = \neg{(x \land y)}$}

\begin{center}
\begin{tabular}{c | c || c | c}
    $x$ & $y$ & $x \lor y$ & $\neg(x \land y)$\\
    \hline
    0 & 0 & 0 & 1\\
    0 & 1 & 1 & 1\\
    1 & 0 & 1 & 1\\
    1 & 1 & 1 & 0
\end{tabular}
\end{center}

\vspace{10px}

Nachdem die Funktionswerte in den jeweiligen Zeilen nicht übereinstimmen ist die Aussage unwahr.

\vspace{10px}

\subsection{$x \lor (y \land z) = (x \lor y) \land (x \lor z)$}

\begin{center}
\begin{tabular}{c | c | c || c | c}
    $x$ & $y$ & $z$ & $x \lor (y \land z)$ & $(x \lor y) \land (x \lor z)$\\
    \hline
    0 & 0 & 0 & 0 & 0\\
    0 & 0 & 1 & 0 & 0\\
    0 & 1 & 0 & 0 & 0\\
    0 & 1 & 1 & 1 & 1\\
    1 & 0 & 0 & 1 & 1\\
    1 & 0 & 1 & 1 & 1\\
    1 & 1 & 0 & 1 & 1\\
    1 & 1 & 1 & 1 & 1
\end{tabular}
\end{center}

\vspace{10px}

Nachdem die Funktionswerte in den jeweiligen Zeilen übereinstimmen ist die Aussage wahr.

Die Funktion $(x \lor y) \land (x \lor z)$ könnte nach der Regel der Distributivität auch zu ${x \lor (y \land z)}$ umformuliert werden, wodurch das Ergebnis offensichtlich wird.

\subsection{$(\neg{x} \land \neg{y}) \lor (x \land z) \lor (y \land z) = (x \land y) \lor (x \land z)$}

\begin{center}
\begin{tabular}{c | c | c || c | c}
    $x$ & $y$ & $z$ & $(\neg{x} \land \neg{y}) \lor (x \land z) \lor (y \land z)$ & $(x \land y) \lor (x \land z)$\\
    \hline
    0 & 0 & 0 & 1 & 0\\
    0 & 0 & 1 & 1 & 0\\
    0 & 1 & 0 & 0 & 0\\
    0 & 1 & 1 & 1 & 0\\
    1 & 0 & 0 & 0 & 0\\
    1 & 0 & 1 & 1 & 1\\
    1 & 1 & 0 & 1 & 1\\
    1 & 1 & 1 & 1 & 1
\end{tabular}
\end{center}

\vspace{10px}

Nachdem die Funktionswerte in den jeweiligen Zeilen nicht übereinstimmen ist die Aussage unwahr.

\end{document}

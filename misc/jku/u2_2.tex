\documentclass{article}
\usepackage[utf8]{inputenc}

\title{Übung 2}
\author{Laurenz Weixlbaumer, 11804751}
\date{Oktober 2018}

\renewcommand{\arraystretch}{1.3}
\renewcommand\thesubsection{(\alph{subsection})}

\usepackage{mathtools}

\begin{document}

\maketitle

\stepcounter{section} % Start at 2
\section{Binärzahlen}

\subsection{}
Befülle folgende Tabelle mit Binärzahlen. Verwende dafür pro Feld die minimal mögliche Anzahl an Bits.

\begin{center}
\begin{tabular}{ c | c | c | c | c | c | c | c | c | c | c | c }
    -4 & -3 & -2 & -1 & 0 & 1 & 2 & 3 & 4 & 5 & 6 & 7\\
    \hline
    100 & 11 & 10 & 1 & 0 & 1 & 10 & 11 & 100 & 101 & 110 & 111\\
    \hline
    1100 & 111 & 110 & 11 & 00/10 & 01 & 010 & 011 & 0100 & 0101 & 0110 & 0111\\
\end{tabular}
\end{center}

Die Ordnung der Zeilen stimmt mit derer auf dem Übungszettel überein.

\subsection{}
Konvertiere die nicht vorzeichenbehaftete Binärzahl $01001011, 0101_2$ in eine Dezimalzahl. Gib dazu den Lösungsweg an.

$$01001011_{2} = 75_{10}$$

Lösungsweg siehe Aufgabe 1, Basisumwandlungen. Die gesamte Zahl kann nun durch Addition des ganzzahligen Teiles und des Bruchteiles ermittelt werden.

$$01001011, 0101_2 = 75_{10} + 0 * 0.5_{10} + 1 * 0.25_{10} + 0 * 0.125_{10} + 1 * 0.0625_{10} = 75, 3125_{10}$$

\clearpage

\subsection{}

Konvertiere die Dezimalzahl $107, 59375_{10}$ in eine nicht vorzeichenbehaftete Binärzahl. Gib dazu den Lösungsweg an.

$$
107_{10} = 1101011_{2}
$$

Der Lösungsweg für die Ermittlung des Bruchteils folgt.

\begin{center}
\begin{tabular}{ c | c }
    Rechnung & Überlauf\\
    \hline
    $0,59375 * 2 = 1,1875$ & 1\\
    $0,1875 * 2 = 0,375$ & 0\\
    $0,375 * 2 = 0,75$ & 0\\
    $0,75 * 2 = 1,5$ & 1\\
    $0,5 * 2 = 1$ & 1
\end{tabular}
\end{center}

Aus den Überläufen (von oben nach unten gelesen) kann nun die Binärzahl $10011_2$ gelesen werden.

Daraus folgt:

$$
107_{10} + 0,59375_{10} = 1101011_{2} + 0,10011_{2} = 1101011,10011_2
$$

\end{document}

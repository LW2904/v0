\documentclass{article}
\usepackage[utf8]{inputenc}

\title{Übung 2}
\author{Laurenz Weixlbaumer, 11804751}
\date{Oktober 2018}

\renewcommand{\arraystretch}{1.3}
\renewcommand\thesubsection{(\alph{subsection})}

\usepackage{mathtools}

\begin{document}

\maketitle

\section{Basisumwandlungen}
Wandle die Hexadezimalzahl $1EE7_{16}$ auf zwei Varianten in eine Oktalzahl um.

\subsection{Variante 1}
$$
1EE7_{16} = 1_{8} * 20_{8}^3 + 16_{8} * 20_{8}^2 + 16_{8} * 20_{8}^1 + 7_{8} * 20_{8}^0 = 17347_{8}
$$

\subsection{Variante 2}

\begin{center}
\begin{tabular}{c | c | c | c }
    Rechnung in Ausgangsbasis & $Rest_{16}$ & $Rest_{8}$ & Letzte Stelle\\
    \hline
    $1EE7 : 8 = 3DC$ & 7 & 7 & 7\\
    $4DC : 8 = 7B$ & C & 14 & 4\\
    $7B : 8 = F$ & 3 & 3 & 3\\
    $F : 8 = 1$ & 7 & 7 & 7\\
    $1 : 8 = 0$ & 1 & 1 & 1
\end{tabular}
\end{center}
Die letzte Stelle des Restes (als Oktalzahl) von unten nach oben gelesen bildet das Ergebnis, $17347_{8}$.

\end{document}

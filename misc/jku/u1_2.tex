\documentclass{article}
\usepackage[utf8]{inputenc}

\title{Übung 1}
\author{Laurenz Weixlbaumer, 11804751}
\date{Oktober 2018}

\renewcommand{\arraystretch}{1.5}
\renewcommand\thesubsection{(\alph{subsection})}

\usepackage{mathtools}

\begin{document}

\maketitle

\stepcounter{section} % Start at 2
\section{Boolesche Algebra}
Prüfe oder widerlege die folgenden Aussagen und verwende dazu die Regeln und
Gesetze der Booleschen Algebra. Gib dazu bei jedem Beweisschritt die verwendete
Regel oder das verwendete Gesetz an.

\subsection{$\overline{a \land (b \lor c) \lor (\overline{a} \land b)} = (\overline{a} \land \overline{b}) \lor (\overline{b} \land \overline{c})$}

Der Übersicht halber wird die Funktionsgleichung im folgenden in $f_1(a, b, c) = \overline{a \land (b \lor c) \lor (\overline{a} \land b)}$ und $f_2(a, b, c) = (\overline{a} \land \overline{b}) \lor (\overline{b} \land \overline{c})$ geteilt, und vorerst getrennt vereinfacht.

\begin{center}
\begin{tabular}{c | c}
    $f_1$\\
    \hline
    $\overline{a \land (b \lor c) \lor (\overline{a} \land b)}$ & Ausgangsfunktion\\
    $\overline{(a \land b) \lor (a \land c) \lor (\overline{a} \land b)}$ & Distributivgesetz\\
    $\overline{b \land (a \lor \overline{a}) \lor (a \land c)}$ & Distributivgesetz\\
    $\overline{b \lor (a \land c)}$ & Komplementärgesetz
\end{tabular}
\end{center}

Es sei angemerkt, dass in den obigen Beweisschritten das Gesetz der Kommutativität neben anderen ohne expliziten Hinweis angewandt wurde.

\begin{center}
\begin{tabular}{c | c}
    $f_2$\\
    \hline
    $(\overline{a} \land \overline{b}) \lor (\overline{b} \land \overline{c})$ & Ausgangsfunktion\\
    $\overline{(a \lor b)} \lor \overline{(b \lor c)}$ & De Morgansche Regel\\
    $\overline{(a \lor b) \land (b \lor c)}$ & De Morgansche Regel\\
    $\overline{b \lor (a \land c)}$ & Distributivgesetz
\end{tabular}
\end{center}

Damit ist das Zutreffen der Aussage $f_1 = f_2$ bewiesen.

\clearpage

\subsection{$(x \lor y) \land (\overline{x} \lor y) = (\overline{x} \lor y) \land (\overline{x} \lor \overline{y})$}

Wie bereits in \textbf{(a)} wird die Funktionsgleichung auch hier in $f_1(x, y) = (x \lor y) \land (\overline{x} \lor y)$ und $f_2(x, y) = (\overline{x} \lor y) \land (\overline{x} \lor \overline{y})$ geteilt, und getrennt vereinfacht.

\begin{center}
\begin{tabular}{c | c}
    $f_1$\\
    \hline
    $(x \lor y) \land (\overline{x} \lor y)$ & Ausgangsfunktion\\
    $y \lor (x \land \overline{x})$ & Distributivgesetz\\
    $y$ & Komplementärgesetz
\end{tabular}
\end{center}

\begin{center}
\begin{tabular}{c | c}
    $f_2$\\
    \hline
    $(\overline{x} \lor y) \land (\overline{x} \lor \overline{y})$ & Ausgangsfunktion\\
    $\overline{x} \lor (y \land \overline{y})$ & Distributivgesetz\\
    $\overline{x}$ & Komplementärgesetz
\end{tabular}
\end{center}

Nachdem $y \neq \overline{x}$, ist die Aussage $f_1 = f_2$ als falsch widerlegt.

\end{document}
